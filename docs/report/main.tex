\documentclass[runningheads]{llncs}

\usepackage{graphicx}
\usepackage[spanish]{babel}
\usepackage{tikz}

\begin{document}

\title{A DSL to create REST services}

\author{Enrique de la Calle Montilla\and
Jorge Blázquez Saborido}
\authorrunning{F. Author et al.}
\institute{Universidad Complutense de Madrid}

\maketitle

\begin{abstract}
The abstract should briefly summarize the contents of the paper in
150--250 words.

\keywords{First keyword  \and Second keyword \and Another keyword.}
\end{abstract}

\section{Introducción}

% - información sobre REST
%   - un poco del contexto en el que nacen
%   - QUÉ es una API REST
% - introducción a los DSLs
% - por qué hacer un DSL para apis rest

Las \emph{APIs REST} son una arquitectura de software para el diseño
de servicios web que está muy extendida en internet desde cerca de
sus inicios. Con ella podemos diseñar la interfaz que muestra una web
al resto de la red. Esta arquitectura se fundamenta en la arquitectura
cliente-servidor, como se ve en la Figura~\ref{fig:arq-cliente-serv},
en la que el cliente y el servidor se envían mensajes. Concretamente,
en una API REST el cliente envía una petición (\emph{request}) al
servidor que este contesta en forma de respuesta (\emph{response}). Las
peticiones pueden ser de cuatro tipos, que corresponden a las
cuatro operaciones CRUD:

\newcommand\POST{\texttt{POST}}
\newcommand\GET{\texttt{GET}}
\newcommand\PUT{\texttt{PUT}}
\newcommand\DELETE{\texttt{DELETE}}

\newcommand\CREATE{\texttt{CREATE}}
\newcommand\READ{\texttt{READ}}
\newcommand\UPDATE{\texttt{UPDATE}}

\begin{itemize}
    \item \POST: es el equivalente a la operación \CREATE, con el que podemos
        crear un nuevo recurso.
    \item \GET: es el equivalente a la operación \READ, con el que podemos
        leer el contenido de un recurso ya existente.
    \item \PUT: es el equivalente a la operación \UPDATE, con el que podemos
        modificar un recurso que ya existe.
    \item \DELETE: es el equivalente a la operación \DELETE, con el que podemos
        eliminar un recurso.
\end{itemize}

\begin{figure}
\begin{center}
    \scalebox{1.3}{
        \begin{tikzpicture}[node distance = 10em]
            \node[draw, fill = red!20, minimum size = 15mm, circle] (c) {Client};
            \node[draw, fill = blue!20, minimum size = 15mm, circle, right of = c] (s) {Server};
            \draw[-stealth]
              (c.north east)
                to[out = 33, in = 90+66]
              node[above] (request) {\small request}
              node[below] {\tiny JSON/...}
              (s.north west);
            \draw[-stealth]
              (s.south west)
                to[out = 180+33, in = -33]
              node[below] (response) {\small response}
              node[above] {\tiny JSON/...}
              (c.south east);
        \end{tikzpicture}
    }
\end{center}
\caption{Arquitectura cliente servidor}
\label{fig:arq-cliente-serv}
\end{figure}

\section{Método}

% y si lo paso a después de herramientas?

% el procseo ha comenzado con una boceto sobre cómo queríamos que fuese
% nuestro lenguaje y luego han venido unas etapas de implementación
% que han ido modificando ligeramente el lenguaje con las limitaciones
% que encontrábamos en nuestra capacidad de implementar lenguajes. Estas
% etapas han sido:
%
% - Creación del metamodelo en EMF
% - Pasar el metamodelo a XText para que nuestro DSL pueda escribirse
%   en texto.
% - (a parte) crear el proyecto de Acceleo y generar código de Spark
%   Java desde instancias del metamodelo

\section{Herramientas}

% Usamos el framework de Eclipse
% Usamos herramientas dentro de Eclipse:
% - EMF
% - xtext
% - acceleo

\section{Evaluación}

% hablar de las pruebas que le hemos hecho al código

\section{Trabajo relacionado}

% quizá quitarlo xD

\section{Conclusiones}


\begin{thebibliography}{8}
\bibitem{ref_article1}
Author, F.: Article title. Journal \textbf{2}(5), 99--110 (2016)

\bibitem{ref_lncs1}
Author, F., Author, S.: Title of a proceedings paper. In: Editor,
F., Editor, S. (eds.) CONFERENCE 2016, LNCS, vol. 9999, pp. 1--13.
Springer, Heidelberg (2016). \doi{10.10007/1234567890}

\bibitem{ref_book1}
Author, F., Author, S., Author, T.: Book title. 2nd edn. Publisher,
Location (1999)

\bibitem{ref_proc1}
Author, A.-B.: Contribution title. In: 9th International Proceedings
on Proceedings, pp. 1--2. Publisher, Location (2010)

\bibitem{ref_url1}
LNCS Homepage, \url{http://www.springer.com/lncs}. Last accessed 4
Oct 2017
\end{thebibliography}

\end{document}
